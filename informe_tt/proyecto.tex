\documentclass[a4paper,12pt,openany,oneside]{book}
\usepackage[utf8]{inputenc}
\usepackage[spanish]{babel}
\usepackage{graphicx}
\usepackage[left=5cm,right=2cm,top=4cm,bottom=4cm,paperwidth=216mm,paperheight=330mm,pdftex]{geometry}
\usepackage{setspace}
\usepackage{listings}
\usepackage{float}
\usepackage{titlesec}
\usepackage{fancyhdr}
\usepackage{dsfont}

 \renewcommand{\thechapter}{\arabic{chapter}} 
 \titleformat{\chapter}[display] 
 {\bfseries\Large} 
 {\filleft\MakeUppercase{\chaptertitlename} \Huge\thechapter} 
 {4ex} 
 {\titlerule 
 \vspace{2ex}% 
 \filright} 
 [\vspace{2ex}% 
 \titlerule]

\fancypagestyle{plain}{%
\fancyhf{} % clear all header and footer fields
\renewcommand{\headrulewidth}{0pt}
\renewcommand{\footrulewidth}{0pt}
}

\doublespacing

\decimalpoint

\floatstyle{boxed}
\newfloat{codigo}{thp}{lop}
\floatname{codigo}{Caja de Código}

\pagestyle{plain}

\begin{document}

\frontmatter

\thispagestyle{empty}

\begin{center}
\textbf{UNIVERSIDAD TECNOLÓGICA METROPOLITANA\\
ESCUELA DE INFORMÁTICA\\}
\vspace{2cm}
RUIDO Y TEXTURAS PROCEDURALES EN TIEMPO REAL.\\
\vspace{2cm}
\end{center}

\begin{flushright}
\parbox[r]{8cm}{TRABAJO DE TÍTULO PARA OPTAR AL\\ TÍTULO DE INGENÍERO CIVIL EN COMPUTACIÓN\\ MENCIÓN INFORMÁTICA.}
\end{flushright}
\vspace{2cm}
\begin{flushright}
PROFESOR GUÍA: Héctor Pincheira Conejeros.\\
\vspace{1cm}
Matías Alejandro Valdenegro Toro.
\end{flushright}
\vspace{6cm}
\begin{center}
SANTIAGO - 2009
\end{center}

\newpage

\begin{flushright}

\vspace*{20 mm}

Nota: \line(1, 0){140} \\

\vspace*{30 mm}

\line(1, 0){180}\\
Firma y Timbre\\
Autoridad Responsable
\end{flushright}

%\chapter*{}

%\begin{flushright}
%\textit{The only thing necessary for evil to triumph is that good men to do nothing.\\}
%\textit{Lo unico necesario para el triunfo del mal, es que los hombres buenos hagan nada.\\}
%Edmund Burke

%\bigskip 

%\textit{You did not bear the shame\\}
%\textit{You resisted\\}
%\textit{Sacrificing your life\\}
%\textit{For freedom, justice and honor.\\}
%Memorial a la Resistencia Alemana, Berlin.

%\end{flushright}

\chapter*{Resumen}

\thispagestyle{empty}

Resumen.

\chapter*{Abstract}

\thispagestyle{empty}

Abstract.

\chapter*{Agradecimientos}

\thispagestyle{empty}

Agradecimientos.

\addtocontents{toc}{\protect\thispagestyle{fancy}}

\tableofcontents

\addtocontents{lof}{\protect\thispagestyle{fancy}}

\listoffigures

\listof{codigo}{Lista de Cajas de Código}

\include{introduccion}

\mainmatter

\fancypagestyle{plain}{%
\fancyhf{} % clear all header and footer fields
\rhead{\bfseries{\thepage}}
\renewcommand{\headrulewidth}{0pt}
\renewcommand{\footrulewidth}{0pt}}

\pagestyle{fancy}

\lhead{\nouppercase{\bfseries{\rightmark}}}
\rhead{\bfseries{\thepage}}
\cfoot{}
\renewcommand{\headrulewidth}{0.4pt}

\chapter{Antecedentes Generales}

\section{Motivación}

Motivacion.

\newpage

\section{Objetivos Generales y Específicos}

\subsection{Objetivos Generales}

\textit{``Objetivo General.''}

\subsection{Objetivos Específicos}

\begin{enumerate}
 \item Objetivo Especifico 1.
 \item Objetivo Especifico 2.
\end{enumerate}

\section{Alcances y Limitaciones}

Alcances y Limitaciones.

\chapter{Estado del Arte}

Estado del Arte.

\chapter{Capitulo N}

\chapter{Conclusiones y Trabajo Futuro}

\bibliographystyle{these}
%Usar bibliografia con bibtex.
%\bibliography{biblio}

\appendix
\renewcommand{\appendixname}{Anexo}

\chapter{Anexo N}

\end{document}
